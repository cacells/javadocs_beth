\documentclass[11pt,a4paper]{article}

\usepackage{longtable}
\newcommand \bt{\begin{longtable}{p{0.25\textwidth}p{0.74\textwidth}}}
\newcommand \et{\end{longtable}}
 
\usepackage[pdftex,usenames,dvipsnames]{color}

\definecolor{classbg}{rgb}{0.707,0.648,0.586}
\definecolor{fieldbg}{rgb}{0.363,0.641,0.746}
\definecolor{conbg}{rgb}{0.711,0.793,0.836}
\definecolor{descriptbg}{rgb}{0.848,0.918,0.953}

\usepackage[T1]{fontenc}
\renewcommand*\familydefault{\sfdefault}

\newcommand{\hs}{\hspace{0.5cm}}

%environment for indented description
\newenvironment{di}
{\begin{flushright}
\begin{minipage}{0.95\textwidth}
\begin{description}
}
{\end{description}
\end{minipage}
\end{flushright}
}

\usepackage[hmargin=2.5cm,vmargin=2.5cm]{geometry}
\setlength{\parindent}{0.05\textwidth}

\begin{document}

\noindent
\colorbox{classbg}{\parbox{1.0\textwidth}{\Large{Class}}}
\begin{di}
\item[\large{thingInfo}]\qquad\\
Classes are a collection of things that have name, type and description. For instance a variable (field) has a name, type, and may be described. A method has a name, a return type, and may also be described. So, thingInfo is a Class to store a collection of three strings called name, type, and description.
\end{di}
\colorbox{fieldbg}{\parbox{1.0\textwidth}{\Large{Fields}}}\vspace{0.5cm}
\bt
\hs \textbf{name} & \emph{type: String}\\
& \hs Name of the thing\\
\hs \textbf{type} & \emph{type: String}\\
& \hs A type associated with the thing. e.g. if the thing is a field then type might be boolean. If the thing is a method, then the type would be the return type of the method, e.g. void. Type is not used for class things or constructor things.\\
\hs \textbf{descript} & \emph{type: String}\\
& \hs A description of the thing.\\
\et

\noindent\colorbox{conbg}{\parbox{1.0\textwidth}{\Large{Constructors}}}
\begin{di}
\item[{thingInfo()}]\qquad\\
Default constructor sets empty strings
\item[{thingInfo(String allinfo)}]\qquad\\
Constructor to set the fields from a single input string which holds name, type, and description separated by forward slashes. The order of the info must be correct.
\item[{thingInfo(String allinfo,boolean swapit)}]\qquad\\
Constructor to set the fields from a single input string. The boolean swapit determines whether the fields are in standard format (name before type) or in swapped format (type before name).
\item[{thingInfo(String thename,String thetype,String thedescript)}]\qquad\\
Unused constructor
\item[{thingInfo(thingInfo orig)}]\qquad\\
Copy constructor
\end{di}
\colorbox{descriptbg}{\parbox{1.0\textwidth}{\Large{Methods}}}
\begin{di}
\item[{cloneVals(thingInfo orig)}]\emph{Returns void}\\
Copy field values from orig\\
\item[{setValsFromString(String allinfo)}]\emph{Returns void}\\
Set the field values from a forward slash delimited string\\
\item[{setName(String thename)}]\emph{Returns void}\\
Set the name field\\
\item[{setType(String thetype)}]\emph{Returns void}\\
Set the type field\\
\item[{setDescript(String thedescript)}]\emph{Returns void}\\
Set the description field\\
\item[{toString()}]\emph{Returns String}\\
Returns a string for printing\\
\item[{breakItUp(String msg)}]\emph{Returns ArrayList$<$String$>$}\\
Break a string into fields at each slash separator.\\
\end{di}

\end{document}
