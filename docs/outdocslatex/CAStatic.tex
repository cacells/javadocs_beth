\documentclass[11pt,a4paper]{article}

\usepackage{longtable}
\newcommand \bt{\begin{longtable}{p{0.25\textwidth}p{0.74\textwidth}}}
\newcommand \et{\end{longtable}}
 
\usepackage[pdftex,usenames,dvipsnames]{color}

\definecolor{classbg}{rgb}{0.707,0.648,0.586}
\definecolor{fieldbg}{rgb}{0.363,0.641,0.746}
\definecolor{conbg}{rgb}{0.711,0.793,0.836}
\definecolor{descriptbg}{rgb}{0.848,0.918,0.953}

\usepackage[T1]{fontenc}
\renewcommand*\familydefault{\sfdefault}

\newcommand{\hs}{\hspace{0.5cm}}

%environment for indented description
\newenvironment{di}
{\begin{flushright}
\begin{minipage}{0.95\textwidth}
\begin{description}
}
{\end{description}
\end{minipage}
\end{flushright}
}

\usepackage[hmargin=2.5cm,vmargin=2.5cm]{geometry}
\setlength{\parindent}{0.05\textwidth}

\begin{document}

\noindent
\colorbox{classbg}{\parbox{1.0\textwidth}{\Large{Class}}}
\begin{di}
\item[\large{CAStatic}]\qquad\\
extends JFrame implements Runnable, ActionListener. This class contains main. It coordinates all simulation and windowing events.
\end{di}
\colorbox{fieldbg}{\parbox{1.0\textwidth}{\Large{Fields}}}\vspace{0.5cm}
\bt
\hs \textbf{experiment} & \emph{type: CAGridStatic}\\
& \hs the current grid of cells and its activities. This is a 1D experiment\\
\hs \textbf{savedvals} & \emph{type: int[][]}\\
& \hs an array to save values from the time steps of the 1D experiment.\\
\hs \textbf{runCount} & \emph{type: int}\\
& \hs current experiment number\\
\hs \textbf{epsCount} & \emph{type: int}\\
& \hs number of eps files that have been written\\
\hs \textbf{newframe} & \emph{type: int}\\
& \hs blunt instrument for slowing the display\\
\hs \textbf{rand} & \emph{type: Random}\\
& \hs random class instance\\
\hs \textbf{runner} & \emph{type: volatile Thread}\\
& \hs the simulation thread\\
\hs \textbf{backImg1} & \emph{type: Image}\\
& \hs the visualisation of the experiment\\
\hs \textbf{backGr1} & \emph{type: Graphics}\\
& \hs some Java thing that takes the image and makes it part of the display or something\\
\hs \textbf{CApicture} & \emph{type: CAImagePanel}\\
& \hs class used to display the changing image amidst the buttons\\
\hs \textbf{startBtn,writeBtn,paramsBtn,wrapBtn} & \emph{type: JButton}\\
& \hs some self explanatory buttons\\
\hs \textbf{msgBtn} & \emph{type: JTextArea}\\
& \hs an area to hold text. Used to display the current parameters.\\
\hs \textbf{buttonHolder} & \emph{type: JPanel}\\
& \hs a panel or window that holds the four buttons in a grid\\
\hs \textbf{scale} & \emph{type: int}\\
& \hs scale factor both x and y for displaying results\\
\hs \textbf{iterations} & \emph{type: int}\\
& \hs holds the current time step\\
\hs \textbf{gSize} & \emph{type: int}\\
& \hs physical grid size both x and y\\
\hs \textbf{maxCellType} & \emph{type: int}\\
& \hs number of different cell types. 0 is a non-cell, i.e. a space\\
\hs \textbf{maxit} & \emph{type: int}\\
& \hs max number of iterations for a run\\
\hs \textbf{started} & \emph{type: boolean}\\
& \hs has a run started?\\
\hs \textbf{palette} & \emph{type: Colour}\\
& \hs palette of colours so that eps and display colours match\\
\hs \textbf{colorindices} & \emph{type: int[]}\\
& \hs indices of the chosen colours\\
\hs \textbf{nnw} & \emph{type: int}\\
& \hs used for colour repitition\\
\hs \textbf{javaColours} & \emph{type: Color[]}\\
& \hs the colours in Java format\\
\hs \textbf{epsColours} & \emph{type: double[][]}\\
& \hs the colours in eps rgb format\\
\et
\noindent\colorbox{conbg}{\parbox{1.0\textwidth}{\Large{Constructors}}}
\begin{di}
\item[{CAStatic(int size)}]\qquad\\
sets up grid size and window size. initialises windows, buttons, and other variables.
\end{di}
\colorbox{descriptbg}{\parbox{1.0\textwidth}{\Large{Methods}}}
\begin{di}
\item[{setpalette()}]\emph{Returns void}\\
sets up the Java and eps colours using the colour indices\\
\item[{saveCA()}]\emph{Returns void}\\
save values from the current time step\\
\item[{outputEPS()}]\emph{Returns void}\\
make a unique eps filename and call the eps printer\\
\item[{changeParameters()}]\emph{Returns void}\\
changes experiment parameters\\
\item[{changeWrap()}]\emph{Returns void}\\
toggle the toroidal wrapping\\
\item[{drawCA()}]\emph{Returns void}\\
update the graphics image and flag a redraw\\
\item[{start()}]\emph{Returns void}\\
start the thread and update the button status\\
\item[{stop()}]\emph{Returns void}\\
stop the thread and update the button status\\
\item[{actionPerformed(ActionEvent e)}]\emph{Returns void}\\
listens for button or other window activity and processes the requests\\
\end{di}
\begin{di}
\item[{run()}]\emph{Returns void}\\
run the experiment\\
\item[{postscriptPrint(String fileName)}]\emph{Returns void}\\
pretty print the results as an eps file\\
\item[{initilise()}]\emph{Returns void}\\
set up a fresh experiment\\
\item[{main(String args[])}]\emph{Returns void}\\
just kicks things off. can be given a fraction as an argument which will not be used in this program\\
\end{di}

\end{document}
