\documentclass[11pt,a4paper]{article}

\usepackage{longtable}
\newcommand \bt{\begin{longtable}{p{0.25\textwidth}p{0.74\textwidth}}}
\newcommand \et{\end{longtable}}
 
\usepackage[pdftex,usenames,dvipsnames]{color}

\definecolor{classbg}{rgb}{0.707,0.648,0.586}
\definecolor{fieldbg}{rgb}{0.363,0.641,0.746}
\definecolor{conbg}{rgb}{0.711,0.793,0.836}
\definecolor{descriptbg}{rgb}{0.848,0.918,0.953}

\usepackage[T1]{fontenc}
\renewcommand*\familydefault{\sfdefault}

\newcommand{\hs}{\hspace{0.5cm}}

%environment for indented description
\newenvironment{di}
{\begin{flushright}
\begin{minipage}{0.95\textwidth}
\begin{description}
}
{\end{description}
\end{minipage}
\end{flushright}
}

\usepackage[hmargin=2.5cm,vmargin=2.5cm]{geometry}
\setlength{\parindent}{0.05\textwidth}

\begin{document}

\noindent
\colorbox{classbg}{\parbox{1.0\textwidth}{\Large{Class}}}
\begin{di}
\item[\large{ParamVals}]\qquad\\
holds experiment parameters
\end{di}
\colorbox{fieldbg}{\parbox{1.0\textwidth}{\Large{Fields}}}\vspace{0.5cm}
\bt
\hs \textbf{pr,pl,ps} & \emph{type: double}\\
& \hs probabilities of moving\\
\hs \textbf{twoPlaces} & \emph{type: DecimalFormat}\\
& \hs static. for pretty printing of floats with 2 d.p.\\
\hs \textbf{asPercent} & \emph{type: DecimalFormat}\\
& \hs static. for pretty printing of percentages\\
\hs \textbf{nowrap} & \emph{type: boolean}\\
& \hs static concerned with the toroidal wrapping\\
\et
\noindent\colorbox{conbg}{\parbox{1.0\textwidth}{\Large{Constructors}}}
\begin{di}
\item[{ParamVals()}]\qquad\\
sets default probabilities
\item[{ParamVals(BufferedReader input)}]\qquad\\
set parameter values using a buffered input from the console. Used before the program had buttons.
\end{di}
\colorbox{descriptbg}{\parbox{1.0\textwidth}{\Large{Methods}}}
\begin{di}
\item[{SetParamVals()}]\emph{Returns void}\\
set parameter values using popup dialog boxes. The currently used method for setting and resetting parameters.\\
\item[{SetParamVals(double dpr,double dpl)}]\emph{Returns void}\\
manually set parameter values. note: these are not checked for validity\\
\item[{changeProbabilities()}]\emph{Returns int}\\
change probabilities using dialog to console and reading from console but not buffered. Used before the program had buttons.\\
\item[{toString()}]\emph{Returns String}\\
for printing the parameters\\
\item[{filename()}]\emph{Returns String}\\
make the parameters into a string that can be part of a filename\\
\item[{readdouble(BufferedReader input,String question)}]\emph{Returns double}\\
ask a question to console and read input from there using buffered reader\\
\end{di}

\end{document}
